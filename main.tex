\documentclass[12pt]{article}
\usepackage[utf8]{inputenc}

\title{proyecto}
\author{Jose Daniel Rivera }
\date{SEPTIEMBRE 2020}
\usepackage[spanish]{babel}

\begin{document}
\title{TALLER MEMORIA. \\}

\maketitle

%\section{}
\noindent
1)	En informática, la memoria es el dispositivo que retiene, memoriza o almacena datos informáticos durante algún periodo de tiempo. La memoria proporciona una de las principales funciones de la computación moderna: el almacenamiento de información y conocimiento..\\
\noindent
\\
2)	Una computadora trabaja con cuatro tipos de memorias diferentes, que sirven para realizar diversas funciones. Estas son la memoria RAM, la memoria ROM, la memoria SRAM o Caché y la memoria Virtual o de Swap.\\
\noindent
\\
En la RAM se guarda distinto tipo de información, desde los procesos temporales como modificaciones de archivos, hasta las instrucciones que posibilitan la ejecución de las aplicaciones que tenemos instaladas en nuestra PC.\cite{senovilla2005}\\
\noindent
\\
Además de la memoria RAM, las computadoras trabajan con la memoria denominada ROM, Read Only Memory, que como su nombre lo indica se trata de una memoria sólo de lectura, ya que la mayoría de estas memorias no pueden ser modificadas debido a que no permiten su escritura.\\
\noindent
\\ 
3)	Se denomina gestión de memoria al acto de gestionar la memoria de un dispositivo informático. De forma simplificada se trata de proveer mecanismos para asignar secciones de memoria a los programas que las solicitan, y a la vez, liberar las secciones de memoria que ya no se utilizan para que estén disponibles para otros programas.\\
\noindent
\\
¿Por qué se necesita la gestión de memoria?

Para optimizar el espacio y poder cargar o intercambiar los programas que van hacer ejecutados del disco duro a la memoria principal.
El administrador de memoria se encarga de llevar un registro de las partes de la memoria que están en uso y de las que no. Si detecta que hay una parte que ya no está en uso, la libera para poder asignarla a los procesos que la necesiten.
El administrador de memoria proporciona  protección y uso compartido, es decir, facilitar un espacio de memoria para cada proceso y controlar que ninguno de ellos trabaje en zonas de memoria que no le han sido asignados.
Administrar el intercambio entre la memoria principal y el disco en los casos en los que la memoria principal no le pueda dar capacidad a todos los procesos que tienen necesidad de ella.
\\
\noindent
\\
4)	La velocidad determina la rapidez a la que es capaz de trabajar la memoria y afecta, junto con el bus de datos, a su ancho de banda. Una mayor velocidad permite realizar transferencias en menos tiempo. Las operaciones de almacenar, borrar y realmacenar nueva información y datos se completarán más rápidamente, lo que en algunos casos puede marcar una diferencia importante de rendimiento..\\
\noindent \cite{stallings2006,}
\\
.
%\begin{itemize}
    %\item \textbf{}
\bibliographystyle{apalike}
    
\bibliography{referencias}
%\usepackage{natbib}
%\end{itemize}
\end{document}
